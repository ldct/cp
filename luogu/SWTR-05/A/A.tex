\listfiles
\documentclass{article}

\usepackage{amsmath}
\usepackage{amssymb}
\usepackage{mathtools}
\usepackage{listings}

\DeclarePairedDelimiter\floor{\lfloor}{\rfloor}
\DeclarePairedDelimiter\ceil{\lceil}{\rceil}
\DeclareMathOperator{\cl}{cl}
\DeclareMathOperator{\E}{E}
\def\Z{\mathbb{Z}}
\def\N{\mathbb{N}}
\def\R{\mathbb{R}}
\def\Q{\mathbb{Q}}
\def\K{\mathbb{K}}
\def\T{\mathbb{T}}
\def\B{\mathcal{B}}
\def\XX{\mathfrak{X}}
\def\YY{\mathfrak{Y}}
\def\AA{\mathfrak{A}}
\def\ZZ{\mathfrak{Z}}
\def\BB{\mathcal{B}}
\def\UU{\mathcal{U}}
\def\MM{\mathcal{M}}
\def\M{\mathfrak{M}}
\def\l{\lambda}
\def\L{\Lambda}
\def\<{\langle}
\def\>{\rangle}

\usepackage[a4paper,margin=1in]{geometry}

\setlength{\parindent}{0cm}
\setlength{\parskip}{1em}

\title{Grid}
\date{}

\begin{document}
\maketitle

Little A has a grid of size $n \times m$ with a number written in each cell. For ease of description, let the cell in the top left corner be $(1,1)$ and the cell in the bottom right corner $(n,m)$.

Little A can enter any of the cells in the bottommost (i.e., $n$th) row and play the game according to the following rules.

1. Let the first time he enters row $i$ be $(i, r_i)$. If he is at cell $(i, r_i)$, then he can only move left or up. Otherwise he can move left, right or up.

2. He cannot leave the grid, unless he is on the first row; leaving the grid from a cell in the first row ends the game.

Define the score of a game as the sum of the numbers on all the squares through which Little A passes. Little A would like you to help him find the lowest score he can get.

Input: the first lines contains $n, m$. The next $n$ lines contains $m$ integers, the values of the $i$th row.


\subsection*{Subtask 1: $a_{i,j} \le 0$}

For 3 points. The answer is the sum of all the $a_{i,j}$.

\subsection*{Subtask 2: $n, m \le 5$}

Brute force.

\subsection*{Subtask 3: $n = 2$}

We classify solutions by the rightmost selected cell of row 1. Fixing the index $i$ of rightmost selected cell, the solution consists of some subarray of the first row ending at $i$, some subarray of the second row ending at $i$, and some subarray of the second row begining at $i$; all these can be selected independently in $O(m)$ time by precomputing the prefix sum. Hence the total time is $O(m^2)$.

\subsection*{Subtask 4: $n, m \le 90$}

Let $f_{i,l,r}$ be the minimum sum of the first $i$ rows if, at the last row, we select $[l, r]$. We have the recurrence

\[
    f_{i,l,r} = \left( \min_{l \le x \le y \le r} f_{i-1,x,y} \right) + \sum_{j \in [l, r]} a_{ij}
\]

Time complexity: from now on assume $m=n$. There are $O(n^3)$ DP states (because there are 3 variables $i,l,r$) and each state depends on $O(n^2)$ other states (because of the enumeration over $x$ and $y$). Hence this is $O(n^5)$. To reduce it, we can precompute $f'_{i,r} = \min_l f_{i,l,r}$.

\subsection*{Subtask 5: $n, m \le 400$}

For every row our DP computes the answer if we select $l, r$ in the $i$th row. In subtask 4, after finishing the DP for a row, we precompute the optimum selection for fixed $r$, and use that to speed up the computation of the $i+1$st row. For this subtask, we can in addition precompute the optimum selection where $r \in [x, y]$ for each $x,y$. The preprocessing time per row increases to $O(n^2)$, but now the DP recurrence for the next row can be evaluated in $O(1)$. The total time complexity is $O(n^2)$.

\subsection*{Subtask 6: $n, m \le 10^3$}

Use the recurrence $f_{i,j} = min(f_{i, r-1} + a_{i,j}, f_{i-1,r} + v_{i,j})$ where $v_{i,j}$ is the smallest subarray of line $i$ ending at $j$.

\end{document}
